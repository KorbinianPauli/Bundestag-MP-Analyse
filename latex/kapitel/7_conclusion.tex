\section{Conclusion}
\label{Conclusion}

Today, information from a wide variety of fields is publicly available on social media and various other forms of online appearances. Using techniques such as web scraping, the data, frequently text, is readily obtained. In order to extract the information contained within, a proper analysis of large-scale unstructured text is often a central task. Within this field, topic modeling plays an important role.

In this paper we applied the \textit{Structural Topic Model} to a large data set of Twitter messages and meta-information associated with members of the German Bundestag. In a first step, examining the estimated proportions of topics provided a concise summary of the predominant themes contained within the tweets. Moreover, associating topic proportions with metadata in a descriptive manner enabled us to explore the topical structure along different dimensions, such as time or membership in a political party. Beyond such explorative analysis, we presented a train-test split framework for the STM, recently developed by \cite{egami2018make}, in order to determine cause-effect relationships between metadata covariates and topics. Extending the traditional topic modeling framework to examine causality between estimated topics and metadata is a challenging task and a current field of research. 

Throughout our analysis we paid particular attention to the statistical assumptions and properties of the STM. While our comparison between the STM and the CTM confirms that metadata does have an influence on the estimated topics, this influence seems to be rather small in general. Nevertheless, we believe that the STM leveraging document-specific characteristics results in an estimated topical structure which is more realistic than is the case when employing models that do not incorporate metadata information. This is also reflected by a higher held-out likelihood of the STM when compared to simpler topic models, as shown by \cite{roberts2016model}. 

When the explicit aim is to investigate the association between metadata and topics, aside from potential improvements in model fit, the advantages of the STM are less obvious. As with other topic models, the estimation of such relationships occurs in a separate second step. That is, the STM (and especially its implementation in the R package \textit{stm}) does not directly produce a usable estimate of the relationship between metadata and topics. Instead, in order to estimate such relations the authors of the STM employ the method of composition, implemented through the function \textit{estimateEffect} in the \textit{stm} package. Within this approach, sampled topic proportions are regressed on metadata covariates using an ordinary least squares (OLS) regression. We have demonstrated several shortcomings of this approach and presented possible alternatives. First, when dealing with (sampled) topic proportions, these are restricted to the interval $(0,1)$. Using regression approaches that assume a dependent variable in $(0,1)$, we extend the method of composition within the framework of the STM. Furthermore, separately modeling topic proportions, as is the case with \textit{estimateEffect}, is a vast simplification, since interdependence among different topics is neglected. We demonstrated this shortcoming by directly assessing the estimated covariance structure of prevalence covariates in section \ref{Direct assessment}, finding that estimated uncertainty increases manifold. In our view, the results obtained by this direct assessment are more realistic than the results achieved using the method of composition. 

When examining causal effects beyond explorative purposes, we suggest to perform a train-test split. Conducting both steps on the same data, i.e., the estimation of topics and the subsequent estimation of effects based on these topics, results in a biased estimation of these effects. As discussed in section \ref{Causal Inference: Train-test Split}, the STM is well-suited for a train-test framework, since it allows for the inclusion of information contained within metadata of the training set when predicting topic proportions on the test set. This is a clear benefit of the STM, emerging from the more advanced design of the STM compared to other topic models such as LDA or CTM.

While we did develop new tools for topic-metadata analysis and address causal inference techniques for topic modeling, this paper has at least two clear limitations which, in turn, constitute areas of future research. First, the direct approach presented in section \ref{Direct assessment} uses MAP estimates of topical prevalence coefficients $\boldsymbol{\Gamma}$; future model implementations should therefore enable sampling from the posterior of $\boldsymbol{\Gamma}$ and thus a fully Bayesian direct assessment. Second, the train-test split in section \ref{Causal Inference: Train-test Split} could potentially be improved, for instance by extracting more of the information available in the metadata. Future work is needed to address such shortcomings and thereby improve the insights generated by topic models.