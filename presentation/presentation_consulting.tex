% !TeX TS-program = pdflatexmk

\documentclass[xcolor=dvipsnames]{beamer}
\usepackage[utf8]{inputenc}
\usepackage[T1]{fontenc}

\usetheme{AnnArbor}
\usepackage{dsfont}
\usepackage{amsmath}
\usepackage{caption}
\usepackage{hyperref}
\usepackage{xcolor}
\usepackage{color}
\usepackage{commath}
\usepackage{physics}
\usepackage{enumerate}

%\usepackage[backend=bibtex, style=authoryear-comp]{biblatex}

\setbeamertemplate{bibliography item}{\insertbiblabel}

\usepackage{filecontents}
\begin{filecontents}{\jobname.bib}
@article{blei2003latent,
  title={Latent dirichlet allocation},
  author={Blei, David M and Ng, Andrew Y and Jordan, Michael I},
  journal={Journal of machine Learning research},
  volume={3},
  number={Jan},
  pages={993--1022},
  year={2003}
}
@article{roberts2016model,
  title={A model of text for experimentation in the social sciences},
  author={Margaret E. Roberts and Brandon M. Stewart and Edoardo M. Airoldi},
  journal={Journal of the American Statistical Association},
  volume={111},
  number={515},
  pages={988--1003},
  year={2016},
  publisher={Taylor \& Francis}
}
\end{filecontents}
\usepackage[style=authoryear]{biblatex}
\renewcommand*{\nameyeardelim}{\addcomma\addspace}
\addbibresource{\jobname.bib}

\newcommand{\customcite}[1]{\citeauthor{#1} (\citeyear{#1})}

\newtheorem{satz}{Satz}

\setbeamertemplate{footline}[page number]

\usecolortheme{seagull}
\setbeamercolor{frametitle}{fg=blue,bg=White}

%\DeclareMathOperator*{\argmax}{arg\, max}
%\DeclareMathOperator*{\argmin}{arg\, min}
%\setbeamertemplate{section in toc}[sections numbered]
%\setbeamertemplate{subsection in toc}[subsections numbered]
%\AtBeginSection[]
%{
%\begin{frame}
%\frametitle{Überblick}
%\tableofcontents[currentsection]
%\end{frame}
%}
%
%\AtBeginSubsection[
% {\frame<beamer>{\frametitle{Überblick}   
%  \tableofcontents[currentsection,currentsubsection]}}%
%]%
%{
 % \frame<beamer>{ 
 %   \frametitle{Überblick}   
   % \tableofcontents[currentsection,currentsubsection]}
%}

\title{Twitter in the Parliament - A Text-based Analysis of German Political Entities}
%\author{Patrick Schulze}
\date{7. Juli 2020}
\author[author1]{Patrick Schulze, Simon Wiegrebe\\[10mm]{\small Supervisors:\\ Prof. Dr. Christian Heumann, Prof. Dr. Paul W. Thurner}}

\begin{document}

\begin{frame}
\titlepage
\end{frame}

%\begin{frame}
%\frametitle{Überblick}
%\tableofcontents[]
%\end{frame}

\section{}
\begin{frame}
\frametitle{Main Contributions}
\begin{itemize}
\item Construction of data set containing more than 500k Tweets and more than 90 variables using web scraping and twitter scraping
\item Topic modeling of German MP Tweets
\item Extension of analytical tools available for examination of topic-metadata relationships 
\item Discussion of causal Inference framework within a topic modeling context
\end{itemize}
\end{frame}

\begin{frame}
\frametitle{Topic Modeling}
\begin{itemize}
\item \textit{Latent Dirichlet Allocation} (LDA) by \textcite{blei2003latent} as first probabilistic topic model
\item Based on LDA and other topic models: \textit{Structural Topic Model} (STM), by \textcite{roberts2016model}
\end{itemize}
\end{frame}

\begin{frame}
\frametitle{Data}
\begin{itemize}
\item MP-level data scraped from \textit{www.bundestag.de/abgeordnete} using BeautifulSoup and Selenium Web Driver
\item Electoral-district-level social-economic data extracted from \textit{www.bundeswahlleiter.de}
\item German federal election 2017 results retrieved from \textit{www.bundeswahlleiter.de}
\item Maximum number of 3200 Tweets per MP downloaded using Tweepy API 
\end{itemize}
\end{frame}

\begin{frame}
\frametitle{Results}
\begin{itemize}
\item Hyperparameter search yields 15 distinct topics
\item Topic labeling conducted manually (human judgment)
\item Descriptive discussion of relationship between metadata and topics
\item Causal Inference: estimation of cause-effect relationships between document-sepcific features (e.g.\ political party) and topics
\end{itemize}
\end{frame}

\begin{frame}
\frametitle{Bibliography}
\printbibliography
\end{frame}
\end{document}